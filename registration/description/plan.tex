\documentclass[paper=a4, fontsize=11pt]{article}
\usepackage{tikz}
\usepackage{siunitx}
\usepackage{pgfplots}
\usepackage{kpfonts}
\usetikzlibrary{shapes, arrows, backgrounds,snakes, positioning}
\usepackage[graphics,tightpage,active]{preview}
\PreviewEnvironment{tikzpicture}
\PreviewEnvironment{equation}
\PreviewEnvironment{equation*}
\newlength{\imagewidth}
\newlength{\imagescale}
\pagestyle{empty}
\thispagestyle{empty}
\title{Plan}
\author{Wenwen Chen}
\date{\normalsize\today}
% These set the width of column and the height of month.
\newcommand{\colWidth}{7cm}
\newcommand{\monthHeight}{4em}
\newcommand{\monthLength}{1.5}
\newcommand{\titleHeight}{2.5}
\newcommand{\leftWidth}{1}
\begin{document}
\tikzset{entry/.style 2 args={
    draw=pgftransparent!0,
    line width=4pt,
    font=\sffamily,
    rectangle,
    rounded corners,
    fill=black!10,
    anchor=north west,
    inner sep=0.2333em,
    text width={\colWidth/#2-1.2em-1.6pt},
    minimum height=#1*\monthHeight,
    align=center
}}
\tikzset{entry1/.style 2 args={
    %xshift=(0.5334em+0.8pt)/2,
    draw=pgftransparent!0,
    line width=4pt,
    font=\sffamily,
    rectangle,
    rounded corners,
    fill=blue!40,
    anchor=north west,
    inner sep=0.2333em,
    text width={\colWidth/#2-1.2em-1.6pt},
    minimum height=#1*\monthHeight,
    align=center
}}
\tikzset{entry2/.style 2 args={
    %xshift=(0.5334em+0.8pt)/2,
    draw=pgftransparent!0,
    line width=4pt,
    font=\sffamily,
    rectangle,
    rounded corners,
    fill=green!30,
    anchor=north west,
    inner sep=0.2333em,
    text width={\colWidth/#2-1.2em-1.6pt},
    minimum height=#1*\monthHeight,
    align=center
}}
\begin{tikzpicture}[y=-\monthHeight,x=\colWidth]

    % First print a list of times.
    \foreach \number/\month in {1.8/2014,2/Feb.,3/Mar.,4/Apr.,5/Mai,6/Jun.,7/Jul.,8/Aug.,9/Sep.}
        \node[anchor=north east] at (\leftWidth,\number*\monthLength) {\month};

    % Draw some day dividers.
		\def\sL{14}
    \draw (\leftWidth-0.01,\titleHeight) -- (\leftWidth-0.01,-\sL*\monthHeight);
		\draw[dashed] (2-0.01,\titleHeight) -- (2-0.01,-\sL*\monthHeight);
    \draw (3-0.01,\titleHeight) -- (3-0.01,-\sL*\monthHeight);
	
		\def\lP{0.85}
		\draw[dashed] (\lP,2*\monthLength+0.07) -- (3+0.01,2*\monthLength+0.07);
		\draw[dashed] (\lP,3*\monthLength+0.07) -- (3+0.01,3*\monthLength+0.07);
		\draw[dashed] (\lP,4*\monthLength+0.07) -- (3+0.01,4*\monthLength+0.07);
		\draw[dashed] (\lP,5*\monthLength+0.07) -- (3+0.01,5*\monthLength+0.07);
		\draw[dashed] (\lP,6*\monthLength+0.07) -- (3+0.01,6*\monthLength+0.07);
		\draw[dashed] (\lP,7*\monthLength+0.07) -- (3+0.01,7*\monthLength+0.07);
		\draw[dashed] (\lP,8*\monthLength+0.07) -- (3+0.01,8*\monthLength+0.07);
		\draw[dashed] (\lP,9*\monthLength+0.07) -- (3+0.01,9*\monthLength+0.07);


    % Start Monday.
    \node[anchor=north] at (1.5,\titleHeight) {Rainer Schoenberger};
    % Write the entries. Note that the x coordinate is 1 (for Monday) plus an
    % appropriate amount of shifting. The y coordinate is simply the starting
    % time.
		\def\mouthL{\monthLength}
    \node[entry={0.4*\mouthL}{0.5}] at (1, 2.7*\mouthL) {Get to know more about experiment};
    \node[entry1={0.4*\mouthL}{1}] at (1, 3.10*\mouthL) {Search Temperature sensor};
    \node[entry={0.4*\mouthL}{0.5}] at (1, 3.50*\mouthL) {Design protocol and Interface};
    \node[entry1={1*\mouthL}{1}] at (1, 3.9*\mouthL) {Read sensor data};
    \node[entry1={0.5*\mouthL}{1}] at (1, 4.9*\mouthL) {Test sensor communication};
    \node[entry1={0.4*\mouthL}{1}] at (1, 5.4*\mouthL) {Construct sensor casing and mounting};
    \node[entry={0.7*\mouthL}{0.5}] at (1, 5.8*\mouthL) {Integration and Test};
    \node[entry={1*\mouthL}{0.5}] at (1, 6.5*\mouthL) {Dokumentation};
    \node[entry={0.3*\mouthL}{0.5}] at (1, 7.5*\mouthL) {Presentation};
    \node[entry={1*\mouthL}{0.5}] at (1, 7.8*\mouthL) {Buffer};

    % The same for Tuesday.
    \node[anchor=north] at (2.5,\titleHeight) {Wenwen Chen};
    \node[entry2={0.4*\mouthL}{1}] at (2.0,3.1*\mouthL) {Search humidity sensor};
    \node[entry2={1*\mouthL}{1}] at (2.0,3.9*\mouthL) {Datatransfer via Ethernet};
    \node[entry2={0.4*\mouthL}{1}] at (2,4.9*\mouthL) {Design sensor casing and mounting};
    \node[entry2={0.5*\mouthL}{1}] at (2,5.3*\mouthL) {Test ethernet communication};
\end{tikzpicture}
\end{document}
